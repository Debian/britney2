\section{excuse.Excuse Class Reference}
\label{classexcuse_1_1Excuse}\index{excuse::Excuse@{excuse::Excuse}}
\subsection*{Public Member Functions}
\begin{CompactItemize}
\item 
def \bf{\_\-\_\-init\_\-\_\-}
\item 
def \bf{set\_\-vers}
\item 
def \bf{set\_\-maint}
\item 
def \bf{set\_\-section}
\item 
def \bf{set\_\-priority}
\item 
def \bf{set\_\-date}
\item 
def \bf{set\_\-urgency}
\item 
def \bf{add\_\-dep}
\item 
def \bf{add\_\-break\_\-dep}
\item 
def \bf{invalidate\_\-dep}
\item 
def \bf{setdaysold}
\item 
def \bf{addhtml}
\item 
def \bf{html}
\end{CompactItemize}
\subsection*{Static Public Attributes}
\begin{CompactItemize}
\item 
tuple \bf{reemail} = re.compile(r\char`\"{}$<$.$\ast$?$>$\char`\"{})\label{classexcuse_1_1Excuse_bb15f55eed8f034db8a64b4ddc46460d}

\begin{CompactList}\small\item\em Regular expression for removing the email address. \item\end{CompactList}\end{CompactItemize}


\subsection{Detailed Description}


\footnotesize\begin{verbatim}Excuse class

This class represents an update excuse, which is a detailed explanation
of why a package can or cannot be updated in the testing distribution from
a newer package in another distribution (like for example unstable).

The main purpose of the excuses is to be written in an HTML file which
will be published over HTTP. The maintainers will be able to parse it
manually or automatically to find the explanation of why their packages
have been updated or not.
\end{verbatim}
\normalsize
 



Definition at line 21 of file excuse.py.

\subsection{Member Function Documentation}
\index{excuse::Excuse@{excuse::Excuse}!__init__@{\_\-\_\-init\_\-\_\-}}
\index{__init__@{\_\-\_\-init\_\-\_\-}!excuse::Excuse@{excuse::Excuse}}
\subsubsection{\setlength{\rightskip}{0pt plus 5cm}def excuse.Excuse.\_\-\_\-init\_\-\_\- ( {\em self},  {\em name})}\label{classexcuse_1_1Excuse_4bdb0917f763d74951c621e466e98bdb}




\footnotesize\begin{verbatim}Class constructor

This method initializes the excuse with the specified name and
the default values.
\end{verbatim}
\normalsize
 

Definition at line 28 of file excuse.py.\index{excuse::Excuse@{excuse::Excuse}!add_break_dep@{add\_\-break\_\-dep}}
\index{add_break_dep@{add\_\-break\_\-dep}!excuse::Excuse@{excuse::Excuse}}
\subsubsection{\setlength{\rightskip}{0pt plus 5cm}def excuse.Excuse.add\_\-break\_\-dep ( {\em self},  {\em name},  {\em arch})}\label{classexcuse_1_1Excuse_60e00fe0515f2dab003bd29baceedd34}




\footnotesize\begin{verbatim}Add a break dependency\end{verbatim}
\normalsize
 

Definition at line 80 of file excuse.py.\index{excuse::Excuse@{excuse::Excuse}!add_dep@{add\_\-dep}}
\index{add_dep@{add\_\-dep}!excuse::Excuse@{excuse::Excuse}}
\subsubsection{\setlength{\rightskip}{0pt plus 5cm}def excuse.Excuse.add\_\-dep ( {\em self},  {\em name})}\label{classexcuse_1_1Excuse_fa97c9f61fef17d6028491362153a766}




\footnotesize\begin{verbatim}Add a dependency\end{verbatim}
\normalsize
 

Definition at line 76 of file excuse.py.\index{excuse::Excuse@{excuse::Excuse}!addhtml@{addhtml}}
\index{addhtml@{addhtml}!excuse::Excuse@{excuse::Excuse}}
\subsubsection{\setlength{\rightskip}{0pt plus 5cm}def excuse.Excuse.addhtml ( {\em self},  {\em note})}\label{classexcuse_1_1Excuse_eb0a1ea0fae66a571e5efa703e53ba3a}




\footnotesize\begin{verbatim}Add a note in HTML\end{verbatim}
\normalsize
 

Definition at line 94 of file excuse.py.\index{excuse::Excuse@{excuse::Excuse}!html@{html}}
\index{html@{html}!excuse::Excuse@{excuse::Excuse}}
\subsubsection{\setlength{\rightskip}{0pt plus 5cm}def excuse.Excuse.html ( {\em self})}\label{classexcuse_1_1Excuse_84049740652a58b248fabdb3fa9d4b2c}




\footnotesize\begin{verbatim}Render the excuse in HTML\end{verbatim}
\normalsize
 

Definition at line 98 of file excuse.py.\index{excuse::Excuse@{excuse::Excuse}!invalidate_dep@{invalidate\_\-dep}}
\index{invalidate_dep@{invalidate\_\-dep}!excuse::Excuse@{excuse::Excuse}}
\subsubsection{\setlength{\rightskip}{0pt plus 5cm}def excuse.Excuse.invalidate\_\-dep ( {\em self},  {\em name})}\label{classexcuse_1_1Excuse_8594c46ccf4182fa8b37fe487bf53850}




\footnotesize\begin{verbatim}Invalidate dependency\end{verbatim}
\normalsize
 

Definition at line 85 of file excuse.py.\index{excuse::Excuse@{excuse::Excuse}!set_date@{set\_\-date}}
\index{set_date@{set\_\-date}!excuse::Excuse@{excuse::Excuse}}
\subsubsection{\setlength{\rightskip}{0pt plus 5cm}def excuse.Excuse.set\_\-date ( {\em self},  {\em date})}\label{classexcuse_1_1Excuse_ac01c3b9802ad26571f01b55ffc1098c}




\footnotesize\begin{verbatim}Set the date of upload of the package\end{verbatim}
\normalsize
 

Definition at line 68 of file excuse.py.\index{excuse::Excuse@{excuse::Excuse}!set_maint@{set\_\-maint}}
\index{set_maint@{set\_\-maint}!excuse::Excuse@{excuse::Excuse}}
\subsubsection{\setlength{\rightskip}{0pt plus 5cm}def excuse.Excuse.set\_\-maint ( {\em self},  {\em maint})}\label{classexcuse_1_1Excuse_189ec1709eef0bd8acb9cd093b8350b5}




\footnotesize\begin{verbatim}Set the package maintainer's name\end{verbatim}
\normalsize
 

Definition at line 56 of file excuse.py.\index{excuse::Excuse@{excuse::Excuse}!set_priority@{set\_\-priority}}
\index{set_priority@{set\_\-priority}!excuse::Excuse@{excuse::Excuse}}
\subsubsection{\setlength{\rightskip}{0pt plus 5cm}def excuse.Excuse.set\_\-priority ( {\em self},  {\em pri})}\label{classexcuse_1_1Excuse_3a0ebe3eb87c1af8f093e80a874ea0fa}




\footnotesize\begin{verbatim}Set the priority of the package\end{verbatim}
\normalsize
 

Definition at line 64 of file excuse.py.\index{excuse::Excuse@{excuse::Excuse}!set_section@{set\_\-section}}
\index{set_section@{set\_\-section}!excuse::Excuse@{excuse::Excuse}}
\subsubsection{\setlength{\rightskip}{0pt plus 5cm}def excuse.Excuse.set\_\-section ( {\em self},  {\em section})}\label{classexcuse_1_1Excuse_6b435fa4d19b929d9fb70c8d28688387}




\footnotesize\begin{verbatim}Set the section of the package\end{verbatim}
\normalsize
 

Definition at line 60 of file excuse.py.\index{excuse::Excuse@{excuse::Excuse}!set_urgency@{set\_\-urgency}}
\index{set_urgency@{set\_\-urgency}!excuse::Excuse@{excuse::Excuse}}
\subsubsection{\setlength{\rightskip}{0pt plus 5cm}def excuse.Excuse.set\_\-urgency ( {\em self},  {\em date})}\label{classexcuse_1_1Excuse_c504d40ac6d07ffdb08b7ff8ed555d10}




\footnotesize\begin{verbatim}Set the urgency of upload of the package\end{verbatim}
\normalsize
 

Definition at line 72 of file excuse.py.\index{excuse::Excuse@{excuse::Excuse}!set_vers@{set\_\-vers}}
\index{set_vers@{set\_\-vers}!excuse::Excuse@{excuse::Excuse}}
\subsubsection{\setlength{\rightskip}{0pt plus 5cm}def excuse.Excuse.set\_\-vers ( {\em self},  {\em tver},  {\em uver})}\label{classexcuse_1_1Excuse_b8751fc5d0033b4c734c476d92841d99}




\footnotesize\begin{verbatim}Set the testing and unstable versions\end{verbatim}
\normalsize
 

Definition at line 51 of file excuse.py.\index{excuse::Excuse@{excuse::Excuse}!setdaysold@{setdaysold}}
\index{setdaysold@{setdaysold}!excuse::Excuse@{excuse::Excuse}}
\subsubsection{\setlength{\rightskip}{0pt plus 5cm}def excuse.Excuse.setdaysold ( {\em self},  {\em daysold},  {\em mindays})}\label{classexcuse_1_1Excuse_cf1fa7c6fb741bbe7e3120113748f3a5}




\footnotesize\begin{verbatim}Set the number of days from the upload and the minimum number of days for the update\end{verbatim}
\normalsize
 

Definition at line 89 of file excuse.py.

The documentation for this class was generated from the following file:\begin{CompactItemize}
\item 
excuse.py\end{CompactItemize}
