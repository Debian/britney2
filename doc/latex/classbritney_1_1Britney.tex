\section{britney.Britney Class Reference}
\label{classbritney_1_1Britney}\index{britney::Britney@{britney::Britney}}
\subsection*{Public Member Functions}
\begin{CompactItemize}
\item 
def \bf{\_\-\_\-init\_\-\_\-}
\item 
def \bf{read\_\-sources}
\item 
def \bf{read\_\-binaries}
\item 
def \bf{read\_\-bugs}
\item 
def \bf{normalize\_\-bugs}
\item 
def \bf{read\_\-dates}
\item 
def \bf{read\_\-urgencies}
\item 
def \bf{read\_\-approvals}
\item 
def \bf{read\_\-hints}
\item 
def \bf{same\_\-source}
\item 
def \bf{get\_\-dependency\_\-solvers}
\item 
def \bf{excuse\_\-unsat\_\-deps}
\item 
def \bf{should\_\-remove\_\-source}
\item 
def \bf{should\_\-upgrade\_\-srcarch}
\item 
def \bf{should\_\-upgrade\_\-src}
\item 
def \bf{reversed\_\-exc\_\-deps}
\item 
def \bf{invalidate\_\-excuses}
\item 
def \bf{write\_\-excuses}
\item 
def \bf{get\_\-nuninst}\label{classbritney_1_1Britney_41c5ee0b9d64b2e76a0c1a1c2b28c73e}

\item 
def \bf{eval\_\-nuninst}\label{classbritney_1_1Britney_e5c18b706e6598474435b1996313cf27}

\item 
def \bf{eval\_\-uninst}\label{classbritney_1_1Britney_b8dd18ead23b6e1126bb4c2a5c3cd8ba}

\item 
def \bf{doop\_\-source}\label{classbritney_1_1Britney_22af61b8a7f6fe71a79d28b5016cae1d}

\item 
def \bf{iter\_\-packages}\label{classbritney_1_1Britney_d453398832baaa7f477f720cfb643029}

\item 
def \bf{do\_\-all}\label{classbritney_1_1Britney_3ef9b6f600eac492fc5aa4b31638198f}

\item 
def \bf{upgrade\_\-testing}
\item 
def \bf{main}
\end{CompactItemize}
\subsection*{Public Attributes}
\begin{CompactItemize}
\item 
\bf{selected}\label{classbritney_1_1Britney_07e771f94d4ce224d5677542db25034d}

\item 
\bf{nuninst\_\-orig}\label{classbritney_1_1Britney_d6169c55786a8ec7fb9773170b652da6}

\end{CompactItemize}
\subsection*{Static Public Attributes}
\begin{CompactItemize}
\item 
tuple \bf{HINTS\_\-STANDARD} = (\char`\"{}easy\char`\"{}, \char`\"{}hint\char`\"{}, \char`\"{}remove\char`\"{}, \char`\"{}block\char`\"{}, \char`\"{}unblock\char`\"{}, \char`\"{}urgent\char`\"{}, \char`\"{}approve\char`\"{})\label{classbritney_1_1Britney_ebbe3f40cca59e2de275b0558556ee63}

\item 
tuple \bf{HINTS\_\-ALL} = (\char`\"{}force\char`\"{}, \char`\"{}force-hint\char`\"{}, \char`\"{}block-all\char`\"{})\label{classbritney_1_1Britney_a088d6fd96963f87f88c9c40cda10bfa}

\end{CompactItemize}


\subsection{Detailed Description}


\footnotesize\begin{verbatim}Britney, the debian testing updater script

This is the script that updates the testing_ distribution. It is executed
each day after the installation of the updated packages. It generates the 
`Packages' files for the testing distribution, but it does so in an
intelligent manner; it try to avoid any inconsistency and to use only
non-buggy packages.

For more documentation on this script, please read the Developers Reference.
\end{verbatim}
\normalsize
 



Definition at line 195 of file britney.py.

\subsection{Member Function Documentation}
\index{britney::Britney@{britney::Britney}!__init__@{\_\-\_\-init\_\-\_\-}}
\index{__init__@{\_\-\_\-init\_\-\_\-}!britney::Britney@{britney::Britney}}
\subsubsection{\setlength{\rightskip}{0pt plus 5cm}def britney.Britney.\_\-\_\-init\_\-\_\- ( {\em self})}\label{classbritney_1_1Britney_5846d81eace24f479292c47e30fd1851}




\footnotesize\begin{verbatim}Class constructor

This method initializes and populates the data lists, which contain all
the information needed by the other methods of the class.
\end{verbatim}
\normalsize
 

Definition at line 201 of file britney.py.\index{britney::Britney@{britney::Britney}!excuse_unsat_deps@{excuse\_\-unsat\_\-deps}}
\index{excuse_unsat_deps@{excuse\_\-unsat\_\-deps}!britney::Britney@{britney::Britney}}
\subsubsection{\setlength{\rightskip}{0pt plus 5cm}def britney.Britney.excuse\_\-unsat\_\-deps ( {\em self},  {\em pkg},  {\em src},  {\em arch},  {\em suite},  {\em excuse} = {\tt None},  {\em excluded} = {\tt []})}\label{classbritney_1_1Britney_f51c60a69f3a9dc2bc5afdb2ffaf3990}




\footnotesize\begin{verbatim}Find unsatisfied dependencies for a binary package

This method analyzes the dependencies of the binary package specified
by the parameter `pkg', built from the source package `src', for the
architecture `arch' within the suite `suite'. If the dependency can't
be satisfied in testing and/or unstable, it updates the excuse passed
as parameter.

The dependency fields checked are Pre-Depends and Depends.
\end{verbatim}
\normalsize
 

Definition at line 709 of file britney.py.\index{britney::Britney@{britney::Britney}!get_dependency_solvers@{get\_\-dependency\_\-solvers}}
\index{get_dependency_solvers@{get\_\-dependency\_\-solvers}!britney::Britney@{britney::Britney}}
\subsubsection{\setlength{\rightskip}{0pt plus 5cm}def britney.Britney.get\_\-dependency\_\-solvers ( {\em self},  {\em block},  {\em arch},  {\em distribution},  {\em excluded} = {\tt []})}\label{classbritney_1_1Britney_5461f49e3e75a251ebedfd37d2a5ff0c}




\footnotesize\begin{verbatim}Find the packages which satisfy a dependency block

This method returns the list of packages which satisfy a dependency
block (as returned by apt_pkg.ParseDepends) for the given architecture
and distribution.

It returns a tuple with two items: the first is a boolean which is
True if the dependency is satisfied, the second is the list of the
solving packages.
\end{verbatim}
\normalsize
 

Definition at line 669 of file britney.py.\index{britney::Britney@{britney::Britney}!invalidate_excuses@{invalidate\_\-excuses}}
\index{invalidate_excuses@{invalidate\_\-excuses}!britney::Britney@{britney::Britney}}
\subsubsection{\setlength{\rightskip}{0pt plus 5cm}def britney.Britney.invalidate\_\-excuses ( {\em self},  {\em valid},  {\em invalid})}\label{classbritney_1_1Britney_171969785db449d7a06c3f762774e0cd}




\footnotesize\begin{verbatim}Invalidate impossible excuses

This method invalidates the impossible excuses, which depend
on invalid excuses. The two parameters contains the list of
`valid' and `invalid' excuses.
\end{verbatim}
\normalsize
 

Definition at line 1112 of file britney.py.\index{britney::Britney@{britney::Britney}!main@{main}}
\index{main@{main}!britney::Britney@{britney::Britney}}
\subsubsection{\setlength{\rightskip}{0pt plus 5cm}def britney.Britney.main ( {\em self})}\label{classbritney_1_1Britney_0e9551bdf927388f55be5ce15a48c94f}




\footnotesize\begin{verbatim}Main method

This is the entry point for the class: it includes the list of calls
for the member methods which will produce the output files.
\end{verbatim}
\normalsize
 

Definition at line 1455 of file britney.py.\index{britney::Britney@{britney::Britney}!normalize_bugs@{normalize\_\-bugs}}
\index{normalize_bugs@{normalize\_\-bugs}!britney::Britney@{britney::Britney}}
\subsubsection{\setlength{\rightskip}{0pt plus 5cm}def britney.Britney.normalize\_\-bugs ( {\em self})}\label{classbritney_1_1Britney_5a6af4a100cfd54e872a27fa7f48ac3c}




\footnotesize\begin{verbatim}Normalize the release critical bug summaries for testing and unstable

The method doesn't return any value: it directly modifies the
object attribute `bugs'.
\end{verbatim}
\normalsize
 

Definition at line 448 of file britney.py.\index{britney::Britney@{britney::Britney}!read_approvals@{read\_\-approvals}}
\index{read_approvals@{read\_\-approvals}!britney::Britney@{britney::Britney}}
\subsubsection{\setlength{\rightskip}{0pt plus 5cm}def britney.Britney.read\_\-approvals ( {\em self},  {\em basedir})}\label{classbritney_1_1Britney_39248f0cfea1c8798b2ca5a97d37eaf8}




\footnotesize\begin{verbatim}Read the approval commands from the specified directory

The approval commands are read from the files contained by the 
`Approved' directory within the directory specified as `basedir'
parameter. The name of the files has to be the same of the
authorized users for the approvals.

The file contains rows with the format:

<package-name> <version>

The method returns a dictionary where the key is the binary package
name followed by an underscore and the version number, and the value
is the user who submitted the command.
\end{verbatim}
\normalsize
 

Definition at line 554 of file britney.py.\index{britney::Britney@{britney::Britney}!read_binaries@{read\_\-binaries}}
\index{read_binaries@{read\_\-binaries}!britney::Britney@{britney::Britney}}
\subsubsection{\setlength{\rightskip}{0pt plus 5cm}def britney.Britney.read\_\-binaries ( {\em self},  {\em basedir},  {\em distribution},  {\em arch})}\label{classbritney_1_1Britney_1b2b0f42e4af1cee472f93e955b30421}




\footnotesize\begin{verbatim}Read the list of binary packages from the specified directory

The binary packages are read from the `Packages_${arch}' files
within the directory specified as `basedir' parameter, replacing
${arch} with the value of the arch parameter. Considering the
large amount of memory needed, not all the fields are loaded
in memory. The available fields are Version, Source, Pre-Depends,
Depends, Conflicts, Provides and Architecture.

After reading the packages, reverse dependencies are computed
and saved in the `rdepends' keys, and the `Provides' field is
used to populate the virtual packages list.

The dependencies are parsed with the apt.pkg.ParseDepends method,
and they are stored both as the format of its return value and
text.

The method returns a tuple. The first element is a list where
every item represents a binary package as a dictionary; the second
element is a dictionary which maps virtual packages to real
packages that provide it.
\end{verbatim}
\normalsize
 

Definition at line 317 of file britney.py.\index{britney::Britney@{britney::Britney}!read_bugs@{read\_\-bugs}}
\index{read_bugs@{read\_\-bugs}!britney::Britney@{britney::Britney}}
\subsubsection{\setlength{\rightskip}{0pt plus 5cm}def britney.Britney.read\_\-bugs ( {\em self},  {\em basedir})}\label{classbritney_1_1Britney_6c777aae69e7bec2efebaf23ddd4a86c}




\footnotesize\begin{verbatim}Read the release critial bug summary from the specified directory

The RC bug summaries are read from the `Bugs' file within the
directory specified as `basedir' parameter. The file contains
rows with the format:

<package-name> <count-of-rc-bugs>

The method returns a dictionary where the key is the binary package
name and the value is the number of open RC bugs for it.
\end{verbatim}
\normalsize
 

Definition at line 406 of file britney.py.\index{britney::Britney@{britney::Britney}!read_dates@{read\_\-dates}}
\index{read_dates@{read\_\-dates}!britney::Britney@{britney::Britney}}
\subsubsection{\setlength{\rightskip}{0pt plus 5cm}def britney.Britney.read\_\-dates ( {\em self},  {\em basedir})}\label{classbritney_1_1Britney_085af5ac906813ea40fc2e623748f517}




\footnotesize\begin{verbatim}Read the upload date for the packages from the specified directory

The upload dates are read from the `Date' file within the directory
specified as `basedir' parameter. The file contains rows with the
format:

<package-name> <version> <date-of-upload>

The dates are expressed as days starting from the 1970-01-01.

The method returns a dictionary where the key is the binary package
name and the value is tuple with two items, the version and the date.
\end{verbatim}
\normalsize
 

Definition at line 483 of file britney.py.\index{britney::Britney@{britney::Britney}!read_hints@{read\_\-hints}}
\index{read_hints@{read\_\-hints}!britney::Britney@{britney::Britney}}
\subsubsection{\setlength{\rightskip}{0pt plus 5cm}def britney.Britney.read\_\-hints ( {\em self},  {\em basedir})}\label{classbritney_1_1Britney_46d535f617fcf1faaaf5d841ea23c184}




\footnotesize\begin{verbatim}Read the hint commands from the specified directory

The hint commands are read from the files contained by the `Hints'
directory within the directory specified as `basedir' parameter. 
The name of the files has to be the same of the authorized users
for the hints.

The file contains rows with the format:

<command> <package-name>[/<version>]

The method returns a dictionary where the key is the command, and
the value is the list of affected packages.
\end{verbatim}
\normalsize
 

Definition at line 580 of file britney.py.\index{britney::Britney@{britney::Britney}!read_sources@{read\_\-sources}}
\index{read_sources@{read\_\-sources}!britney::Britney@{britney::Britney}}
\subsubsection{\setlength{\rightskip}{0pt plus 5cm}def britney.Britney.read\_\-sources ( {\em self},  {\em basedir})}\label{classbritney_1_1Britney_054f44c47f17c0c4f5a069e821b7f868}




\footnotesize\begin{verbatim}Read the list of source packages from the specified directory

The source packages are read from the `Sources' file within the
directory specified as `basedir' parameter. Considering the
large amount of memory needed, not all the fields are loaded
in memory. The available fields are Version, Maintainer and Section.

The method returns a list where every item represents a source
package as a dictionary.
\end{verbatim}
\normalsize
 

Definition at line 292 of file britney.py.\index{britney::Britney@{britney::Britney}!read_urgencies@{read\_\-urgencies}}
\index{read_urgencies@{read\_\-urgencies}!britney::Britney@{britney::Britney}}
\subsubsection{\setlength{\rightskip}{0pt plus 5cm}def britney.Britney.read\_\-urgencies ( {\em self},  {\em basedir})}\label{classbritney_1_1Britney_09fc27899506b4830b1961f125a7b6a4}




\footnotesize\begin{verbatim}Read the upload urgency of the packages from the specified directory

The upload urgencies are read from the `Urgency' file within the
directory specified as `basedir' parameter. The file contains rows
with the format:

<package-name> <version> <urgency>

The method returns a dictionary where the key is the binary package
name and the value is the greatest urgency from the versions of the
package that are higher then the testing one.
\end{verbatim}
\normalsize
 

Definition at line 509 of file britney.py.\index{britney::Britney@{britney::Britney}!reversed_exc_deps@{reversed\_\-exc\_\-deps}}
\index{reversed_exc_deps@{reversed\_\-exc\_\-deps}!britney::Britney@{britney::Britney}}
\subsubsection{\setlength{\rightskip}{0pt plus 5cm}def britney.Britney.reversed\_\-exc\_\-deps ( {\em self})}\label{classbritney_1_1Britney_be1b4af9d6c6650c70b24267412bc1a8}




\footnotesize\begin{verbatim}Reverse the excuses dependencies

This method returns a dictionary where the keys are the package names
and the values are the excuse names which depend on it.
\end{verbatim}
\normalsize
 

Definition at line 1099 of file britney.py.\index{britney::Britney@{britney::Britney}!same_source@{same\_\-source}}
\index{same_source@{same\_\-source}!britney::Britney@{britney::Britney}}
\subsubsection{\setlength{\rightskip}{0pt plus 5cm}def britney.Britney.same\_\-source ( {\em self},  {\em sv1},  {\em sv2})}\label{classbritney_1_1Britney_85d2e45e8431779b62f398c34972ddf1}




\footnotesize\begin{verbatim}Check if two version numbers are built from the same source

This method returns a boolean value which is true if the two
version numbers specified as parameters are built from the same
source. The main use of this code is to detect binary-NMU.
\end{verbatim}
\normalsize
 

Definition at line 630 of file britney.py.\index{britney::Britney@{britney::Britney}!should_remove_source@{should\_\-remove\_\-source}}
\index{should_remove_source@{should\_\-remove\_\-source}!britney::Britney@{britney::Britney}}
\subsubsection{\setlength{\rightskip}{0pt plus 5cm}def britney.Britney.should\_\-remove\_\-source ( {\em self},  {\em pkg})}\label{classbritney_1_1Britney_f8a6c9adbdec7a5a982dd2b74febcc08}




\footnotesize\begin{verbatim}Check if a source package should be removed from testing

This method checks if a source package should be removed from the
testing distribution; this happen if the source package is not
present in the unstable distribution anymore.

It returns True if the package can be removed, False otherwise.
In the former case, a new excuse is appended to the the object
attribute excuses.
\end{verbatim}
\normalsize
 

Definition at line 765 of file britney.py.\index{britney::Britney@{britney::Britney}!should_upgrade_src@{should\_\-upgrade\_\-src}}
\index{should_upgrade_src@{should\_\-upgrade\_\-src}!britney::Britney@{britney::Britney}}
\subsubsection{\setlength{\rightskip}{0pt plus 5cm}def britney.Britney.should\_\-upgrade\_\-src ( {\em self},  {\em src},  {\em suite})}\label{classbritney_1_1Britney_94785175a85f44b1afaf3add167a211f}




\footnotesize\begin{verbatim}Check if source package should be upgraded

This method checks if a source package should be upgraded. The analisys
is performed for the source package specified by the `src' parameter, 
checking the architecture `arch' for the distribution `suite'.
       
It returns False if the given package doesn't need to be upgraded,
True otherwise. In the former case, a new excuse is appended to
the the object attribute excuses.
\end{verbatim}
\normalsize
 

Definition at line 903 of file britney.py.\index{britney::Britney@{britney::Britney}!should_upgrade_srcarch@{should\_\-upgrade\_\-srcarch}}
\index{should_upgrade_srcarch@{should\_\-upgrade\_\-srcarch}!britney::Britney@{britney::Britney}}
\subsubsection{\setlength{\rightskip}{0pt plus 5cm}def britney.Britney.should\_\-upgrade\_\-srcarch ( {\em self},  {\em src},  {\em arch},  {\em suite})}\label{classbritney_1_1Britney_bd18d7acde434387e94344a39db5b0e5}




\footnotesize\begin{verbatim}Check if binary package should be upgraded

This method checks if a binary package should be upgraded; this can
happen also if the binary package is a binary-NMU for the given arch.
The analisys is performed for the source package specified by the
`src' parameter, checking the architecture `arch' for the distribution
`suite'.
       
It returns False if the given package doesn't need to be upgraded,
True otherwise. In the former case, a new excuse is appended to
the the object attribute excuses.
\end{verbatim}
\normalsize
 

Definition at line 789 of file britney.py.\index{britney::Britney@{britney::Britney}!upgrade_testing@{upgrade\_\-testing}}
\index{upgrade_testing@{upgrade\_\-testing}!britney::Britney@{britney::Britney}}
\subsubsection{\setlength{\rightskip}{0pt plus 5cm}def britney.Britney.upgrade\_\-testing ( {\em self})}\label{classbritney_1_1Britney_921effe0d64ed713c38888684a0a26d4}




\footnotesize\begin{verbatim}Upgrade testing using the unstable packages

This method tries to upgrade testing using the packages from unstable.
\end{verbatim}
\normalsize
 

Definition at line 1438 of file britney.py.\index{britney::Britney@{britney::Britney}!write_excuses@{write\_\-excuses}}
\index{write_excuses@{write\_\-excuses}!britney::Britney@{britney::Britney}}
\subsubsection{\setlength{\rightskip}{0pt plus 5cm}def britney.Britney.write\_\-excuses ( {\em self})}\label{classbritney_1_1Britney_010f6deffca32f7f71ecf1f5c1bb4985}




\footnotesize\begin{verbatim}Produce and write the update excuses

This method handles the update excuses generation: the packages are
looked to determine whether they are valid candidates. For the details
of this procedure, please refer to the module docstring.
\end{verbatim}
\normalsize
 

Definition at line 1154 of file britney.py.

The documentation for this class was generated from the following file:\begin{CompactItemize}
\item 
britney.py\end{CompactItemize}
